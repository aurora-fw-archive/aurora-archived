\part{Getting Started}
\chapter{Aurora Overview}
\section{Introduction}
Aurora is a powerful, scalable and cross platform framework for general purpose. It works on UNIX-like platforms, Windows, OS X, Android, iOS, in a lot of microprocessors and also on some standalone platforms. Aurora is released under the Aurora Public License (AuPL), which allows use the framework with a flexible licensing. Aurora base architecture is developed under C++ programming language. Bindings for many other languages have been written, including C, Python and Java.

\section{Features}
\begin{itemize}
\item Cross-Platform Support: Aurora allow developers to develop application for multiple targets, easily.
\item Flexible License: With Aurora Public License (AuPL) developers don't need to worry about copyrights infringement and other concerns which not allow a flexible licensing.
\item Languages Binding: Aurora is highly compatible with other languages like C, Python, Java, ...
\item Powerful Wrappers: Developers can be more productive, scalable and flexible with improved OOPL (Object Oriented Programming Language) Wrappers.
\item Cryptography Algorithms: Aurora have the fastest algorithms for Military Cryptography and Hashing with wide range of algorithms since standardized to the most powerful.
\item Amazing Graphical Engine: Aurora have an amazing graphical engine that allow developers to use the last OpenGL API, Vulkan API or DirectX (only on Windows Platforms) implementations.
\item Network Sockets: Aurora comes with fast network socket implementation that adapts to diferent platforms.
\end{itemize}

\chapter{Installation}
The installation procedure is different on each Aurora platform. This chapter provide information on how to install Aurora, as well as software and hardware requirements for using Aurora on each of the supported platforms. Please follow the instructions for your platform from this following sections and subsections.
\section{Windows}
\section{UNIX-like Platforms}
\subsection{Linux}

\section{MacOS X}
\section{Microprocessors}